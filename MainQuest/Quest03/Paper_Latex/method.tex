\section{Method}

본 연구에서는 자율주행 데이터를 활용하여 도로 결함을 탐지하는 U-Net과 U-Net++ 모델을 구현하였으며, 두 가지 세그멘테이션 접근법으로 이진 분류와 다중 클래스 세그멘테이션을 비교하였다. 주요 설정은 다음과 같다.

\subsection{모델 구조 및 분류 설정}
\begin{itemize}
    \item \textbf{U-Net}: 인코더-디코더 구조로 전역적인 특징을 추출하며, 스킵 커넥션을 통해 정보 손실을 줄인다. 이진 분류 세그멘테이션에서는 결함 여부를 예측하고, 다중 분류 세그멘테이션에서는 다양한 결함 유형을 예측한다.
    \item \textbf{U-Net++}: 네스티드 스킵 커넥션을 통해 다중 해상도의 특징을 통합하여 더욱 정밀한 결함 탐지가 가능하다. 이진 분류와 다중 분류 모두에서 결함의 세밀한 탐지를 목표로 한다.
\end{itemize}

\subsection{손실 함수 설정}
손실 함수는 모델 학습의 방향성을 결정하며, 본 연구에서는 이진 분류와 다중 분류에 대해 각각 다른 손실 함수를 사용하였다.
\begin{itemize}
    \item \textbf{이진 분류 세그멘테이션}:
    \begin{itemize}
        \item U-Net: 이진 교차 엔트로피 (Binary Cross-Entropy, BCE)
        \item U-Net++: Dice 손실
    \end{itemize}
    \item \textbf{다중 클래스 세그멘테이션}:
    \begin{itemize}
        \item U-Net: 범주형 교차 엔트로피 (Categorical Cross-Entropy)
        \item U-Net++: 범주형 교차 엔트로피와 Dice 손실의 조합
    \end{itemize}
\end{itemize}

각 손실 함수는 다음과 같이 정의된다:

\[
L_{\text{BCE}} = - \frac{1}{N} \sum_{i=1}^{N} \left( y_i \log(\hat{y}_i) + (1 - y_i) \log(1 - \hat{y}_i) \right)
\]

\[
L_{\text{Dice}} = 1 - \frac{2 \sum_{i=1}^{N} y_i \hat{y}_i}{\sum_{i=1}^{N} y_i + \sum_{i=1}^{N} \hat{y}_i}
\]

\[
L_{\text{Categorical}} = - \frac{1}{N} \sum_{i=1}^{N} \sum_{c=1}^{C} y_{i,c} \log(\hat{y}_{i,c})
\]

\[
L_{\text{U-Net++}} = \alpha L_{\text{Categorical}} + (1 - \alpha) L_{\text{Dice}}
\]

여기서 $y_i$는 실제 레이블, $\hat{y}_i$는 예측값을 나타내며, $\alpha$는 두 손실 함수의 가중치를 조절하는 파라미터이다.

\subsection{평가 메트릭}
평가 메트릭으로는 Dice 손실을 포함한 모델(U-Net++의 다중 분류)에서는 **Dice 계수**와 **Road IoU**를 사용하였다. Dice 손실이 없는 모델(U-Net의 이진 분류 및 다중 분류)에서는 **Road IoU**만을 사용하였다. 각각의 메트릭은 다음과 같이 정의된다:

\[
\text{Dice Coefficient} = \frac{2 \sum_{i=1}^{N} y_i \hat{y}_i}{\sum_{i=1}^{N} y_i + \sum_{i=1}^{N} \hat{y}_i}
\]

\[
\text{IoU} = \frac{\sum_{i=1}^{N} y_i \hat{y}_i}{\sum_{i=1}^{N} \left( y_i + \hat{y}_i - y_i \hat{y}_i \right)}
\]

특히 Road IoU 메트릭은 다중 클래스 세그멘테이션에서도 사용되어, 각 클래스별로 도로 결함 탐지 성능을 세밀하게 평가하였다.
