\section{Introduction}

자율주행 기술의 발전은 우리의 일상과 교통 인프라를 빠르게 변화시키고 있다. 도로 위에서 발생하는 결함은 운전자에게는 불편을 초래하고 자율주행 차량에게는 안전상의 위협을 가하는 주요 요인이므로 결함을 사전에 감지하고 신속하게 대응하는 것이 필수적이다. 자율주행 차량과 같은 지능형 모빌리티 장치에는 고해상도 카메라, 라이더(LiDAR), 센서 등이 장착되어 있어 다양한 환경 조건에서 실시간으로 도로 환경을 모니터링할 수 있다. 이들은 도로의 상태 정보를 대량으로 축적하고 있으며 이러한 데이터는 자율주행 기술뿐 아니라 도로 유지보수, 인프라 관리 등 폭넓은 분야에서 유용하게 활용될 수 있다.

최근 연구들에서 자율주행 차량에서 수집한 데이터를 도로 결함 탐지뿐만 아니라 저공 비행 드론에 활용하여 인프라 점검에 활용될 수 있음을 제시하였다. 드론은 접근하기 어려운 지역에서도 쉽게 사용할 수 있고 빠르고 효율적으로 넓은 영역을 모니터링할 수 있는 장점이 있다. 이러한 드론 기반의 도로 및 인프라 점검은 도로 결함이나 구조물 손상의 사전 탐지 및 관리를 가능하게 하여 사회적 비용을 줄이고 안전성을 높이는 데 기여할 수 있다. 

기존의 전통적인 이미지 처리 기법은 단순한 결함을 탐지하는 데는 어느 정도 유효하지만 실제 도로 환경의 복잡한 조건을 반영하지 못한다. 조명 변화, 날씨, 다양한 도로 구조로 인해 전통적인 방식은 실제 응용에서 신뢰성 있는 결과를 제공하지 못한다. 따라서 최근에는 딥러닝 기반의 세그멘테이션 모델이 등장하여 픽셀 단위의 높은 정확도로 도로 결함을 탐지하는 데 사용되고 있다. 특히 Fully Convolutional Network (FCN)와 같은 초기 모델부터 인코더-디코더 구조를 기반으로 하는 U-Net과 그 확장 모델인 U-Net++ 등이 세그멘테이션 분야에서 성공적으로 적용되고 있다.

본 연구는 자율주행 데이터를 활용하여 도로 결함을 탐지하고 저공 비행 드론의 인프라 점검 응용으로 확장할 가능성을 탐구한다. 단순히 도로 여부를 구분하는 이진 분류 세그멘테이션 접근법을 넘어 다양한 유형을 구분할 수 있는 다중 분류 세그멘테이션 접근법을 도입하여 결함 탐지의 세분화와 정확도를 높이고자 한다. 다중 분류 세그멘테이션 접근법은 도로 여부 파악뿐만 아니라 결함 크기, 결함의 치명적 여부를 세분화하여 자율주행 차량이 단순 회피만이 아닌 상황에 따라 적절히 대응할 수 있도록 돕는다. 이를 저공 비행 드론에 활용할 시 세분화된 정보를 통해 유지보수 인력이 우선적으로 점검해야 할 구역을 사전에 파악하는 데 유용하다.

본 연구는 U-Net과 U-Net++ 모델을 활용하여 자율주행 데이터를 기반으로 한 이진 분류 및 다중 분류 세그멘테이션 실험을 수행하여, 각 모델의 복잡도와 성능을 비교한다. 또한 데이터 증강과 손실 함수의 최적화가 도로 결함 탐지 성능에 미치는 영향을 분석한다. 특히 U-Net에서는 이진 분류에 binary cross-entropy를 다중 분류에서는 categorical cross-entropy를 적용한다. U-Net++에서는 각 작업에 대해 dice 손실을 추가하여 성능을 보완한다. 평가 metric으로 dice 계수와 'Road IoU' 를 사용하여 다중 클래스 세그멘테이션에서도 정밀한 평가를 수행하며, 도로 결함 탐지에 있어 다중 분류 접근법의 효용성을 강조한다.

본 연구는 자율주행 데이터를 활용한 도로 결함 탐지가 기존의 이진 분류에서 다중 분류 세그멘테이션으로 확장될 때의 성능 변화를 평가하며 자율주행 차량뿐 아니라 드론 기반의 인프라 점검에도 응용 가능성을 넓히는 데 그 목적이 있다. 자율주행 데이터를 다양한 상황에서 안정적으로 활용할 수 있는지에 대한 새로운 방향성을 제시한다. 향후 연구에서는 이러한 다중 분류 접근법을 기반으로 실제 응용 가능한 경량화된 세그멘테이션 모델을 개발하는 것을 목표로 한다.
