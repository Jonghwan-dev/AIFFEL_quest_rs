\section{Related work}

도로 결함 탐지와 관련된 초기 연구들은 전통적인 컴퓨터 비전 기법에 기반했다. 경계 검출(edge detection), 색상 분할(color segmentation), 텍스처 분석(texture analysis) 등은 단순한 결함을 탐지하는 데는 어느 정도 유효하지만, 환경 변화에 민감하여 실제 응용에서는 성능이 크게 저하되었다. 특히 조명 조건이나 날씨, 도로 구조의 차이는 이러한 방법들의 신뢰성을 떨어뜨리는 주요 요인이었다.

딥러닝의 도입으로 Fully Convolutional Network (FCN), SegNet, U-Net과 같은 모델들이 제안되었다. FCN은 최초의 픽셀 단위 예측이 가능한 모델로, 도로 결함 탐지에도 적용되었다. SegNet은 인코더-디코더 구조를 통해 이미지의 디테일을 보존하여 보다 정밀한 세그멘테이션을 가능하게 한다. U-Net은 인코더-디코더 구조에 스킵 커넥션을 추가하여, 정보 손실을 줄이고 세밀한 결함 탐지가 가능하다. U-Net++는 nested skip connection을 통해 다양한 해상도에서 도출된 특징을 통합하여 복잡한 결함을 탐지하는 데 적합한 모델이다.

본 연구는 자율주행 데이터를 활용하여 U-Net과 U-Net++ 모델을 기반으로 도로 결함을 탐지하고, 다중 분류 세그멘테이션 접근을 통해 모델의 성능을 향상시키는 방법을 제안한다.
