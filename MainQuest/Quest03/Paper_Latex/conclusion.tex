\section{Conclusion}

본 연구에서는 자율주행 데이터를 사용한 도로 결함 탐지에서 이진 분류와 다중 분류 세그멘테이션 접근을 비교하였다. U-Net++ 모델은 다중 클래스 세그멘테이션에서 더 높은 성능을 보여주었으며, Dice 손실을 추가하여 불균형 클래스 문제를 개선하였다. 이 연구는 자율주행 데이터가 저공 비행 드론의 인프라 점검에도 유용하게 활용될 수 있음을 시사하며, 향후 경량화된 모델과 최적화된 손실 함수를 통해 실시간 응용 가능성을 높이고자 한다.

